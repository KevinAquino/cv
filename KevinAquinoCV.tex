%%%%%%%%%%%%%%%%%%%%%%%%%%%%%%%%%%%%%%%%%
% Friggeri Resume/CV
% XeLaTeX Template
% Version 1.0 (5/5/13)
%
% !TEX TS-program = xelatex
% This template has been downloaded from:
% http://www.LaTeXTemplates.com
%
% Original author:
% Adrien Friggeri (adrien@friggeri.net)
% https://github.com/afriggeri/CV
%
% License:
% CC BY-NC-SA 3.0 (http://creativecommons.org/licenses/by-nc-sa/3.0/)
%
% Important notes:
% This template needs to be compiled with XeLaTeX and the bibliography, if used,
% needs to be compiled with biber rather than bibtex.
%
%%%%%%%%%%%%%%%%%%%%%%%%%%%%%%%%%%%%%%%%%

\documentclass[print]{friggeri-cv} % Add 'print' as an option into the square bracket to remove colors from this template for printing

\addbibresource{bibliography.bib} % Specify the bibliography file to include publications
\usepackage{url}
\newcommand{\ka}{\textbf{Aquino, K.M.}}
\newcommand{\mb}{Breakspear, M.}
\newcommand{\pr}{Robinson, P.A.}
\newcommand{\mms}{{Schira, M.M.}}
\newcommand{\pd}{Drysdale, P.M.}
\newcommand{\jp}{Pang, J.C.}
\newcommand{\tl}{Lacy, T.C.}
\newcommand{\stf}{Francis, S.T.}
\newcommand{\kjm}{Mullinger, K.J.}
\newcommand{\rs}{Sokoliuk, R.}
\newcommand{\sm}{Mayhew, S.}
\newcommand{\sh}{Hanslmayr, S.}
\newcommand{\rsp}{Sanchez, R.}
\newcommand{\ohbm}{Annual Meeting of the Organization for Human Brain Mapping}
\newcommand{\sfn}{Annual meeting for the Society for Neuroscience}
\usepackage{fancyhdr}
\usepackage{amsmath}
\pagestyle{fancy}
\renewcommand{\headrulewidth}{0pt}
\rfoot{Page \thepage}
\cfoot{}

%\pagestyle{fancy}
\begin{document}

\header{Dr Kevin}{Aquino}{BSc PhD} % Your name and current job title/field
\setcounter{page}{1}

%
%
%\begin{minipage}[t]{0.5\textwidth}
%
%Complex Systems Group \\
%School of Physics A28, University of Sydney \\
%NSW 2006, Australia
%
%%  410 Arps Hall \\
%%  1945 N High Street \\
%%  Columbus, OH 43210
%\end{minipage}
%\begin{minipage}[t]{0.5\textwidth}
%{\bf M: } +61 421 851 157 \\
%{\bf Ph: }  +61 2 8627 0662 \\
%%  Fax: (614) 292-3906 \\
%{\bf E: } aquino@physics.usyd.edu.au \\
%{\bf W: }~\url{http://www.physics.usyd.edu.au/~aquino/}
%\end{minipage}


%\begin{minipage}[t]{0.5\textwidth}
%  Sir Peter Mansfield Imaging Center \\
%  School of Physics \& Astronomy, \\ University of Nottingham \\
%  Nottingham NG7 2RD, United Kingdom

\begin{minipage}[t]{0.5\textwidth}
  Brain, Mind and Society Research Hub \\
  Monash Institute of Cognitive \& Clinical \\
  Neurosciences (MICCN), Monash University \\
  770 Blackburn Rd, Clayton, 3168 Vic, Australia
  
%  410 Arps Hall \\
%  1945 N High Street \\
%  Columbus, OH 43210
\end{minipage}
%\begin{minipage}[t]{0.5\textwidth}
%\parbox[t]{2cm}{\textbf {Mobile:}\\ \textbf{Phone:}\\ \textbf{Email:}\\ \textbf{Homepage:}}
%\parbox[t]{8cm}{ +44 (0) 7 534 299 034\\+44 (0) 115 9514747 \\
%kevin.aquino@nottingham.ac.uk \\
%~\url{https://kevinaquino.github.io}%{http://www.physics.usyd.edu.au/~aquino/}}
%}
\begin{minipage}[t]{0.5\textwidth}
\parbox[t]{2cm}{\textbf {Mobile:}\\ \textbf{Phone:}\\ \textbf{Email:}\\ \textbf{Homepage:}}
\parbox[t]{8cm}{ +61 (0) 421 851 157 \\+61 (0) 3 9902 9800 \\
kevin.aquino@monash.edu \\
~\url{https://kevinaquino.github.io}%{http://www.physics.usyd.edu.au/~aquino/}}
}

\end{minipage}

%----------------------------------------------------------------------------------------
%	SIDEBAR SECTION
%----------------------------------------------------------------------------------------
%
%\begin{aside} % In the aside, each new line forces a line break
%\section{contact}
%A28 School of Physics
%University of Sydney, NSW 2006
%Australia
%~
%{\bf M: }+61 (2) 0421 851 157
%~
%\href{mailto:aquino@physics.usyd.edu.au}{{\bf E: }aquino@\\physics.usyd.edu.au}
%\section{research interests}
%fMRI, Hemodynamics,
%Modeling, Vision, 
%Neural Field Theory,
%Networks, Data analysis.
%\section{programming}
%MATLAB,
%\LaTeX
%(proficient)
%Python,  
%C,
%C++,
%CSS.
%(familiar)
%\section{languages}
%english
%spanish
%\end{aside}

%----------------------------------------------------------------------------------------
% CURRENT EMPLOYMENT
%----------------------------------------------------------------------------------------
%\section{Contact}
%A28 School of Physics\\
%University of Sydney, NSW 2006\\
%Australia\\
%\\
%{\bf M: }+61 (2) 0421 851 157\\
%\\
%\href{mailto:aquino@physics.usyd.edu.au}{{\bf E: }aquino@physics.usyd.edu.au}\\

\vspace{0.5cm}
\section{Current employment}

\begin{entrylist}
\entry
{2017 -- }
{Brain and Mental Health Research Hub}
{MICCN, Monash University, Australia}
{\emph{Research Fellow, Head of Computational Modeling}\\
Employed as a research fellow at Monash University with the school of Psychology (MICCN) under the Brain division of the Brain and Mental health (BMH) research hub lead by Prof. Alex Fornito. This role includes:
\begin{itemize}
\item
Setting up collaborative projects between BMH and Prof Gustavo Deco at the Universitat Pompeu Fabra, Barcelona. 
\item
Employing and developing physiological models to understand resting state fMRI networks
\item
Being the Head of Computational Modeling with the Brain division of BMH research Hub
\item
Supervising Honours, and PhD students within BMH
\end{itemize}
%{2015 --}
%{Sir Peter Mansfield Imaging Center}
%{University of Nottingham, Nottingham UK}
%{\emph{Research Fellow}\\
%Employed as a research fellow position on a Leverhulme Trust funded grant ``Revealing the origin of human alpha oscillations using ultra high-field fMRI-EEG'' with Prof. Susan Francis, Dr. Karen Mullinger, \& Dr. Rosa Sanchez-Panchuelo.
%This includes the following roles:
%\begin{itemize}
%\item
%Optimising MRI sequences at 7T to uncover the cortical lamination of visual cortex {\it in vivo}.
%\item
%Optimising fMRI sequences at 7T to measure the functional responses with combined EEG.
%\item
%Create a pipeline to optimize understanding layer-responses.
%\item
%Apply spatiotemporal hemodynamic models to analyse imaging data
%\item
%Design, collection and analyses of fMRI experiments in 7T and 3T scanner for various projects.
%\end{itemize}
}
\end{entrylist}


%----------------------------------------------------------------------------------------
%	EDUCATION SECTION
%----------------------------------------------------------------------------------------


\section{Education}

\begin{entrylist}
\entry
{2012}
{Doctor of Philosophy (Physics)}
{}
{University of Sydney\\
Thesis: \emph{ Spatiotemporal Hemodynamics: From Theory to Experiment} \\ (December) Supervised by Prof. Peter Robinson (principal), Prof. Michael Breakspear (associate), Dr Mark Schira (associate) and Dr Peter Drysdale (associate).}
%------------------------------------------------
\entry
{2006}
{Bachelor of Science (Honors I)}
{}
{University of Sydney,\\
Thesis: \emph{Wavelet analysis of Brain activity} \\ Supervised by Prof. Peter Robinson (principal), Prof. Michael Breakspear (associate).}
%------------------------------------------------
\entry
{2005}
{Bachelor of Science}
{}
{University of Sydney,\\Majors in Physics and Mathematics.}
%    University of Sydney, 2005.
%\end{itemize}
\end{entrylist}

\section{Expertise}

\textbf{Research areas} fMRI, MRI, Spatiotemporal Hemodynamics, Modeling, Neural Field Theory, Data analysis, Image processing, Deconvolution, task-fMRI, \& resting-state fMRI. \\
%\textbf{Mathematical modeling} 
\textbf{Programming:} Proficient in MATLAB, \LaTeX \& Python. Familiar in C, C++, CSS, \& Java.\\
\textbf{Software:} Freesurfer, SPM, FSL, AFNI, mrTools, mrVista. \\

%----------------------------------------------------------------------------------------
%	WORK EXPERIENCE SECTION
%----------------------------------------------------------------------------------------
\newpage
\section{Research experience}

\begin{entrylist}

%------------------------------------------------
\entry
{2015 -- 2017}
{University of Nottingham}
{Nottingham, Nottinghamshire, UK}
{\emph{Research Fellow, July 2015 -- July 2017}\\
Worked with Prof. Susan Francis, Dr Karen Mullinger, and Rosa Sanchez-Panchuelo on Laminar fMRI, Ultra-high field neuroimaging and EEG-fMRI at 7T.}

\entry
{2013 -- 2015}
{University of Sydney}
{Camperdown, NSW, Australia}
{\emph{Research Associate, Janurary 2013 -- July 2015}\\
Worked with Prof. Peter Robinson and Dr. Mark Schira on projects with Spatiotemporal hemodynamics, Brain Connectivity, Neural Field Modelling and co-supervised 3 PhD Students.}
\entry
{2012 -- 2013}
{Queensland Institute of Medical Research}
{Herston, QLD, Australia}
{\emph{Research Assistant, July 2012 -- January 2013}\\
Worked under Part time (0.5) with Prof. Michael Breakspear projects on Deconvolution of fMRI data.}
\entry
{2012 -- 2013}
{University of Sydney}
{Camperdown, NSW, Australia}
{\emph{Research Associate, July 2012 -- January 2013}\\
Worked under Part time (0.5) with Prof. Peter Robinson on projects with Spatiotemporal hemodynamics.
}
\entry
{2005 -- 2006}
{The Blackdog institute}
{Randwick, NSW, Australia}
{Research Assistant to Prof. Michael Breakspear\\
Casual work to analyse Functional Magnetic Resonance Imaging (fMRI) data, and electrical encephalography (EEG). Also worked on a neural mass model that resulted in a publication.
}
\end{entrylist}

%----------------------------------------------------------------------------------------
%	AWARDS SECTION
%----------------------------------------------------------------------------------------

\section{Honors and awards}
\begin{entrylist}
\entry
{2020}
{Australian Research Council Discovery Project}
{}
{Awarded \$509,561 AUD For project: Modelling brain network development (DP200103509): with Alex Fornito, Gustavo Deco, and Kevin Aquino (CI-C). }
\entry
{2017}
{PLOS Early Career Travel Award Program}
{}
{Awarded \$500 USD to attend ISMRM 2017 in Hawaii see: http://blogs.plos.org/thestudentblog/2017/08/08/how-science-is-communicated-impacts-how-science-is-received-early-career-researchers-share-their-take-on-the-future-of-scicomm/}

\entry
{2016}
{Guarantors of Brain Travel Grant}
{}
{Guarantors of Brain Travel Grant to attend the Organization for Human Brain Mapping Annual Meeting in Geneva, Switzerland travel Award, \textsterling 600 GBP.}
\entry
{2013}
{Trainee Abstract Travel Award}
{}
{Organization for Human Brain Mapping Annual Meeting in Seattle, WA, USA travel Award for abstract: ``Spatiotemporal neural dynamics from fMRI: Deconvolution with a spatiotemporal HRF'', \$700 USD. }
\entry
{2007 -- 2010}
{Australian Postgraduate Award Scholarship}
{}
{Australian government award. Stipend for \$20,000 AUD  p/a while taking a research degree.}
\entry
{2006}
{Science Foundation for Physics, Scholarship No.~3}
{}
{Faculty of Science, University of Sydney Prize for \$1000 AUD.}
\entry
{2005}
{Summer student scholarship}
{}
{Neuroscience Institute of Schizophrenia and Allied Disorders (NISAD) award: \$2000 AUD, for work on ``Multiscale character of evoked cortical activity'' with Prof. Michael Breakspear.}
\entry
{2003 -- 2005}
{Faculty of Science Dean's List of Excellence in Academic Performance}
{}
{University of Sydney award for Years 1--3 of the undergraduate degree for having an average of High Distinction (85/100) or greater.}
\entry
{2003 -- 2004}
{Talented Student Program}
{}
{University of Sydney first and second year program for students to undertake in projects in various research groups. First year: electron microscopy unit (Vicki Keast), Second year: Complex Systems (Michael Breakspear).}
\end{entrylist}

%----------------------------------------------------------------------------------------
%	PUBLICATIONS SECTION
%----------------------------------------------------------------------------------------
\section{Publications}


\begin{enumerate}
\item
\paper{\ka, Fulcher B.F., Parkes, L., Sabaroedin, K., \& Fornito, A.}{2019}{Identifying and removing widespread signal deflections from fMRI data: Rethinking the global signal regression problem}{bioRxiv}{}{\url{https://www.biorxiv.org/content/10.1101/662726v1}}
\item
\paper{\ka, \rs, Pakenham, D., \rsp, \sh, \sm, \kjm, \& \stf}{2019}{Addressing challenges of high spatial resolution, UHF fMRI for group analysis of higher-order cognitive tasks; an inter-sensory task directing attention between visual and somatosensory domains.}{Human Brain Mapping}{40}{1298--1316}
\item
\paper{\rs, \sm, \ka, Wilson, R., Brookes, M.J., \stf, \sh, \& \kjm}{2019}{Two spatially distinct posterior alpha sources fulfill different functional roles in attention}{Journal of Neuroscience}{39}{7183--7194}
\item
\paper{Parkes, L., Tiego, J., \ka, Braganza, L., Chamberlain, S.R., Fontenelle, L., Harrison, B.J., Lorenzetti, V., Paton, B., Razi, A., Fornito, A., Yucel, M.}{2019}{Transdiagnostic variations in impulsivity and compulsivity in obsessive-compulsive disorder and gambling disorder correlate with effective connectivity in cortical-striatal-thalamic-cortical circuits.}{Neuroimage}{202}{116070}
\item
\paper{Tewarie, P.K, Hunt B.A.E, O'Neill, G.C., Byrne, A., \ka, Bauer, M., Mullinger, K.J., Coombes, S.C., \& Brookes, M.J.}{2018}{Relationships between neuronal oscillatory amplitude and dynamic functional connectivity}{Cerebral Cortex}{}{1--14}
\item
\paper{\jp, \ka, \pr, \tl, \& \mms}{2017}{Biophysically based method to deconvolve spatiotemporal neurovascular signals from fMRI data}{J. Neurosci. Meth.}{308}{6--20}
\item
\paper{Pandejee, G.M., \pr, Henderson, J.A., \ka, \& Sarkar, S.}{2017}{Inference of Direct and Multistep Effective Connectivities from Functional Connectivity of the Brain and of Relationships to Cortical Geometry}{J. Neurosci. Meth.}{283}{42-54}
\item
\paper{\jp, \pr, \ka, \& N., Vasan.}{2017}{Effects of astrocytic dynamics on spatiotemporal hemodynamics: Modeling and enhanced data analysis}{Neuroimage}{147}{994--1005}
\item
\paper{\tl, \ka, \& \pr}{2016}{Shock-like BOLD Responses Induced in the Primary Visual Cortex by Moving Visual Stimuli}{J. R. Soc. Interface}{13 (125)}{20160576}
\item
\paper{\pr, Zhao, X., \ka, Griffiths, J.D., Sarkar, S., \& Pandejee, G.M}{2016}{Eigenmodes of Brain Activity: Neural Field Theory Predictions and Comparison with Experiment}{Neuroimage}{142}{79--89}
\item
\paper{\jp, \pr, \& \ka}{2016}{Response-mode decomposition of spatio-temporal haemodynamics}{J. R. Soc. Interface}{13 (118)}{20160253}
\item
\paper{Puckett, A., \ka, \pr, \mb, \& \mms}{2016}{The 3D hemodynamic response function for depth-dependent fMRI of human cortex}{Neuroimage}{139}{240--248}
\item
\paper{\ka, \pr, \mms, \& \mb}{2014}{Deconvolution of neural dynamics from fMRI data using a spatiotemporal hemodynamic response function}{Neuroimage}{94}{203--215}
\item
\paper{\ka, \pr, \& \pd}{2014}{Spatiotemporal hemodynamic response functions derived from physiology}{Journal of Theoretical Biology}{347}{118--136}
\item
\paper{\ka, \mms, \pr, \pd, \& \mb}{2012}{Hemodynamic traveling waves in Human Visual Cortex}{PLoS - Computational Biology}{8}{e1002435}
\item
\paper{\pd,  Huber, J.P., \pr, \& \ka}{2010}{Spatiotemporal BOLD hemodynamics from a poroelastic hemodynamic model}{The journal of Theoretical Biology}{265}{523--534}
\item
\paper{Freyer, F., \ka, \pr, Ritter P., \& \mb}{2009}{Bistability and non-Gaussian fluctuations in spontaneous cortical activity}{The Journal of Neuroscience}{29}{8512--8524}
\item
\paper{\mb, Bullmore, Ed T., \ka, Das, P., \& Williams, L. M.}{2006}{The multiscale character of evoked cortical activity} {Neuroimage}{30}{1230--1242}
\end{enumerate}

%----------------------------------------------------------------------------------------
%	Oral presentations
%----------------------------------------------------------------------------------------
%\newpage
\section{Oral presentations}
\begin{enumerate}

\item
\oral{\ka}{April 2020}{Hemodynamic modelling of fMRI time signals}{Educational workshop: 28th Annual meeting for the International Society of Magnetic Resonance Imaging in Medicine}{Sydney, Australia} 

\item
\oral{\ka}{October 2019}{Educational workshop: Pre-processing of MRI: State-of-the-art, challenges and pitfalls}{\ohbm: Australian Chapter.}{Newcastle, Australia} 

\item
\oral{\ka}{June 2019}{OHBM}{More than Meets the Eye: Methods for characterizing and removing large-scale structured signals from resting-state fMRI data.}{Morning workshop: The Global Signal Strikes Back: Understanding and Addressing Widespread Signal Fluctuations in fMRI 25th \ohbm}{Rome, Italy} 

\item
\oral{\ka}{March 2019}{More than meets the eye: methods for characterizing and removing large- scale structured signals from resting-state fMRI}{Noosa workshop on Brain Function, Connectivity and Behaviour}{Noosa, Australia} 

\item
\oral{\ka}{November, 2018}{On the intersection between theory and experiment in large- scale brain network modelling}{NeuroEng}{Sydney, Australia} 

\item
\oral{\ka, \pr, Puckett, A., \mb, \& \mms}{June 2014}{From Visual Stimulus to BOLD Measurements, a complete spatiotemporal model derived from submillimetre fMRI}{Morning workshop: The hemodynamic response and neurovascular coupling: From sources to measures to models, 20th \ohbm}{Hamburg, Germany} 

\item
\oral{\ka, \pr, \mms, Lacy, T., \& \mb}{November 2013}{Spatiotemporal hemodynamics from a physiological model: Deconvolution of fMRI data, and interactions of BOLD responses}{Meeting for the Society for Neuroscience}{San Diego, CA, USA} 

\item
\oral{\ka, \pr, \mms, \& \mb}{June 2013}{Spatiotemporal neural dynamics from fMRI: Deconvolution with a spatiotemporal HRF}{19th \ohbm}{Seattle WA, USA} (Recipient of the trainee abstract travel award).

\item
\oral{\ka, \mms, \pr, \& \mb}{February 2013}{A physiologically plausible spatiotemporal model of BOLD allows deconvolution of hemodynamic and neuronal response components}{ANS 2013: the 33rd Annual Meeting of the Australian Neuroscience Society}{Melbourne, VIC, Australia}

\item
\oral{\ka}{December 2012}{Spatiotemporal Neural Dynamics from fMRI: Deconvolution using a spatiotemporal hemodynamic response function}{Brain Modes, Large-scale models of the brain}{Queensland Institute of Medical Research, QLD, Australia}

\item
\oral{\ka, \mms, \pr, \pd, \& \mb}{December 2011}{Spatiotemporal Hemodynamics: From theory to Experiment}{5th Australian Workshop on Computational Neuroscience}{University of Western Sydney, NSW, Australia}
\end{enumerate}

\section{Poster presentations}

\begin{enumerate}
\item
\poster{\ka, Sanchez-Panchuelo, R. S., Mullinger, K.J., Hanslmayr, S., Mayhew S., Sokoliuk, R., \& Francis, S.T.}{April 2017}{Top-down modulation in a directed sensory attention task}{25th Annual meeting for the International Society of Magnetic Resonance Imaging in Medicine}{5368}{Honolulu, HI, USA.}

\item
\poster{\ka, Sanchez-Panchuelo, R. S., Mullinger, K.J., Hanslmayr, S., Mayhew S., Sokoliuk, R., \& Francis, S.T.}{June 2016}{In vivo segmentation of layer IV in primary visual cortex using anatomy}{22nd \ohbm}{1256}{Geneva, Switzerland}

\item
\poster{\tl, \ka, \pr,\& \mms}{June 2015}{Induction of Hemodynamic Shocks by Moving Stimuli in the Primary Visual Cortex}{21st \ohbm}{3739}{Honolulu, Hawaii, USA}

\item
\poster{\ka, \tl, \pr,\& \mms}{June 2015}{Using models to design fMRI experiments -- not just fit data}{21st \ohbm}{3729}{Honolulu, Hawaii, USA}

\item
\poster{Puckett, A. M., \ka, Isherwood, Z.,\& \mms}{June 2015}{Laminar Differences in Retinotopic Maps: Measured and Modeled}{21st \ohbm}{4049}{Honolulu, Hawaii, USA}

\item
\poster{\pr, Zhao, X., \ka, Griffiths, J.D., Sarkar, S., \& Pandejee G.M}{June 2015}{Neural Field Theory of Eigenmodes of Brain Activity}{21st \ohbm}{3870}{Honolulu, Hawaii, USA}

\item
\poster{Puckett, A., \ka, \pr, \mb, \& \mms}{June 2014}{Laminar analysis: The spatiotemporal profile of the BOLD response changes with depth}{20th \ohbm}{2062}{Hamburg, Germany}

\item
\poster{\ka, \pr, Puckett, A., \mb \& \mms}{June 2014}{From Visual Stimulus to BOLD Measurements, a complete spatiotemporal model derived from sub millimetre fMRI}{20th \ohbm}{1702}{Hamburg, Germany}

\item
\poster{\ka, \pr, \mms, \& \mb}{June 2013}{Spatiotemporal neural dynamics from fMRI: Deconvolution with a spatiotemporal HRF}{19th \ohbm}{1752}{Seattle, WA, USA} (Recipient of the trainee abstract travel award).

\item
\poster{\ka, \mb, \pr, \& \mms}{November 2011}{Disambiguating between neural and hemodynamic effects}{\sfn}{619.12/YY4}{Washington, DC, USA}

\item
\poster{\mms, \mb, Paxinos, G., \& \ka}{November 2011}{High resolution characterisaton of spatiotemporal point spread of BOLD: Vascular versus neuronal spread in human V1 and V2}{\sfn}{271.14/JJ5}{Washington, DC, USA}

\item
\poster{\ka, \pd, \& \mb}{June 2010}{A spatiotemporal HRF derived from physiology}{16th \ohbm}{2456}{Barcelona, Spain}

\item
\poster{\pd, Huber J.P., \pr, \& \ka}{June 2010}{Spatiotemporal BOLD hemodynamics from a poroelastic hemodynamic model}{16th \ohbm}{1080}{Barcelona, Spain}

\item 
\poster{\ka, \mms, \pd, \pr, \& \mb}{June 2010}{BOLD travelling waves in primary visual cortex}{16th \ohbm}{2460}{Barcelona, Spain}

%\item
%\poster{\ka, \pr, \mb, \pd, \& \mms}{December 2010}{Traveling waves in spatiotemporal hemodynamics}{19th Australian Institute of Physics Congress}{}{Melbourne, VIC, Australia.}

\item
\poster{\ka, \pr, \mms, \pd, \& \mb}{June 2008}{Characterization of physiologic and neural fluctuations in sensory-evoked fMRI of the primary visual cortex}{14th \ohbm}{491}{Melbourne, VIC, Australia}

\end{enumerate}
\newpage
\section{Teaching \& supervision}
\begin{entrylist}
\entry
{2019}
{PhD Primary supervisor}
{}
{Supervising student YuChi at Monash University.}
\entry
{2018 --}
{Honours Primary supervisor}
{}
{Supervised Honours students Anna Earl,Alex Wulff, and Jia Lee in their 4th year psychology projects at Monash University.}
\entry
{2017 --}
{Associate supervisor}
{}
{Associate supervisor to PhD candidates: Kristina Sabaroedin, Sid Chopra, and Ashlea Segal}
\entry
{2016}
{Undergraduate Teaching}
{}
{Assisted in teaching 4th year undergraduate physics module : Image processing at the University of Nottingham. Involved in marking student presentations as well as creating and supervising 2 student projects (2 students per project).}
\entry
{2016}
{Modeling Seminar Series}
{}
{Co-organized and gave 2 Lectures of a Modeling seminar series at the Sir Peter Mansfield Imaging center. Six lectures were scheduled with an attendance of \approx 20 members that included staff from around the Univeristy of Nottingham.}
\entry
{2013 --}
{Associate Supervisor}
{}
{Honors Student (4th year) Thomas Lacy (2013), PhD Candidate Thomas Lacy (from 2014), PhD Candidate James Pang (from 2015, submitted 2018), and PhD candidate Grishma Mehta-Pandejee (completed 2018)}
\entry
{2014}
{First year informatics course: INFO 1903}
{}
{Guest Lecture: "Hemodynamics: from Theory to experiment."}
\entry
{2011 -- 2013}
{Supervisor}
{}
{Third year Physics special project students: Thomas Lacy (2011) \& Nikhil Vasan (2013).}
\entry
{2011 -- 2012}
{Teaching Assistant}
{}
{Redesign of Computational Physics module in Physics II.}
\entry
{2010 -- 2012}
{Laboratory Supervisor}
{}
{\href{http://www.physics.usyd.edu.au/current/units_offered_yr2.shtml}{Physics II}, Computational Physics.}
\entry
{2006 -- 2009}
{University Tutor}
{}
{Computational Physics II, Physics I, and Computational Science (First year course).}
\end{entrylist}


\printbibsection{article}{article in peer-reviewed journal} % Print all articles from the bibliography

\printbibsection{book}{books} % Print all books from the bibliography

\begin{refsection} % This is a custom heading for those references marked as "inproceedings" but not containing "keyword=france"
\nocite{*}
\printbibliography[sorting=chronological, type=inproceedings, title={international peer-reviewed conferences/proceedings}, notkeyword={france}, heading=subbibliography]
\end{refsection}

\begin{refsection} % This is a custom heading for those references marked as "inproceedings" and containing "keyword=france"
\nocite{*}
\printbibliography[sorting=chronological, type=inproceedings, title={local peer-reviewed conferences/proceedings}, keyword={france}, heading=subbibliography]
\end{refsection}

\printbibsection{misc}{other publications} % Print all miscellaneous entries from the bibliography

\printbibsection{report}{research reports} % Print all research reports from the bibliography

%----------------------------------------------------------------------------------------
%----------------------------------------------------------------------------------------
%	INTERESTS SECTION
%----------------------------------------------------------------------------------------
%\newpage
\section{Academic Service}
\begin{enumerate}
\item
Organized an educational course to teach Dynamic Causal modelling and ordinary differential equations. Lectures were made online at \url{https://www.youtube.com/watch?v=q-yypnHgCII} and have reached over 100 hours of viewing time.
\item
Gave pre-processing seminars at the Sir Peter Mansfield Imaging Centre, University of Nottingham (2016,2018) attended by $\approx 50$ staff and student members. 
\item
Organizer of Complex systems seminar series. (2013--2015)
\item
Webmaster, Complex Systems group \url{http://physics.usyd.edu.au/complex-systems/} (2015)
\item
Reviewer: Neuroimage, Neural Networks), Frontiers in Neuroscience, Human Brain Mapping (Conference Abstracts), Human Brain mapping, PLoS Computational Biology, Journal of Neuroscience Methods, \& Science Advances. 
\item
Volunteer for University of Sydney Open day.
\end{enumerate}
%
%\section{References}
%\begin{referenceList}
%\reference
%{Prof. Peter Robinson}
%{robinson@physics.usyd.edu.au}
%{+61-02-9351-3779}
%{School of Physics, University of Sydney.}
%\reference
%{Dr Mark Schira}
%{mark.schira@gmail.com}
%{+ 61-02-4239-2501}
%{School of Psychology, University of Wollongong.}
%\reference
%{Prof. Michael Breakspear}
%{mjbreaks@gmail.com }
%{+61-07-3845-3692}
%{QIMR Berghofer Medical Research Institute.}
%\end{referenceList}


%\section{Interests, et cetera}
%\textbf{Personal:} snorkeling, graphic design (see \url{http://www.physics.usyd.edu.au/~aquino/Site/Et_cetera.html})
%%{http://www.physics.usyd.edu.au/~aquino/Site/Et\_cetera.html})
%, visual arts, cooking, single malt whiskey.\\
%{\bf Blog:} \url{http://sciencelifeetc.blogspot.com.au}\\
%{\bf Personal website} \url{http://www.physics.usyd.edu.au/~aquino/}

\end{document}